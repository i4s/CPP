\documentclass[10pt,pdf,hyperref={unicode}]{beamer}
% \documentclass[aspectratio=43]{beamer}
% \documentclass[aspectratio=1610]{beamer}
% \documentclass[aspectratio=169]{beamer}

\usepackage{lmodern}

% подключаем кириллицу 
\usepackage[T2A]{fontenc}
\usepackage[utf8]{inputenc}
\usepackage{listings}
\usepackage{textcomp}
% \usepackage[T1]{fontenc}

% отключить клавиши навигации
\setbeamertemplate{navigation symbols}{}

% тема оформления
\usetheme{CambridgeUS}

% цветовая схема
\usecolortheme{lily}

\title{КПП}   
\subtitle{Кроссплатформенное программирование}
\author{\tiny https://github.com/i4s}
\date{\tiny \today} 
\logo{\includegraphics[height=15mm]{images/logo.jpg}\vspace{-7pt}}

\begin{document}

% титульный слайд
\begin{frame}
\titlepage
\end{frame} 

\begin{frame}
\frametitle{Введение в курс} 
\framesubtitle{Процесс проведения занятий}
\begin{enumerate}
  \item 8 общих занятий
  \begin{enumerate}
    \item вопросы по предыдущему материалу
    \item немного теории
    \item приём лабораторных работ
  \end{enumerate}
  \item 8 занятий в подгруппах
  \begin{enumerate}
    \item получение лабораторных заданий
    \item практика изученной теории
  \end{enumerate}
  \item "А \underline{ходить} надо?"(с)
\end{enumerate}
\end{frame}

\begin{frame}
\frametitle{Введение в курс} 
\framesubtitle{Правила игры просты...}
\begin{enumerate}
  \item 8 лабораторных работ
  \begin{enumerate}
    \item 4 - Java (Февраль-Март)
    \item 4 - Scala (Апрель-Май)
  \end{enumerate}
  \item лабораторные индивидуальные
  \item работы будут опубликованы
  \item учёт ведётся по срокам сдачи:
  \begin{enumerate}
    \item сдано на паре-выдаче: 3
    \item на следующей: 2
    \item ещё позже: 1
    \item и так далее
  \end{enumerate}
\end{enumerate}
\end{frame}

\begin{frame}
\frametitle{Переменные и типы данных}
\framesubtitle{Простые типы}
\begin{itemize}
  \item<1-> Варианты?
  \item<2-> Целые числа
  \begin{enumerate}
    \item byte
    \item short
    \item int
    \item long
  \end{enumerate}
  \item<3-> Числа с плавающей точкой
  \begin{enumerate}
    \item float
    \item double
  \end{enumerate}
  \item<4-> Символы
  \begin{enumerate}
    \item char
  \end{enumerate}
  \item<5-> Логические значения
  \begin{enumerate}
    \item boolean
  \end{enumerate}
\end{itemize}
\end{frame}
%\lstset{
%  basicstyle=\footnotesize\tt,   % the size of the fonts that are used for the code
%  breakatwhitespace=false,       % sets if automatic breaks should only happen at whitespace
%  breaklines=true,               % sets automatic line breaking
%  captionpos=b,                  % sets the caption-position to bottom
%  extendedchars=true,            % lets you use non-ASCII characters; for 8-bits encodings only, does not work with UTF-8
%  frame=single,                  % adds a frame around the code
%  language=Java,                 % the language of the code
%  keywordstyle=\bf,
%  showspaces=false,              % show spaces everywhere adding particular underscores; it overrides 'showstringspaces'
%  showstringspaces=false,        % underline spaces within strings only
%  showtabs=false,                % show tabs within strings adding particular underscores
%  tabsize=2                      % sets default tabsize to 2 spaces
%}
\definecolor{javared}{rgb}{0.6,0,0} % for strings
\definecolor{javagreen}{rgb}{0.25,0.5,0.35} % comments
\definecolor{javapurple}{rgb}{0.5,0,0.35} % keywords
\definecolor{javadocblue}{rgb}{0.25,0.35,0.75} % javadoc
 
\lstset{language=Java,
basicstyle=\ttfamily,
keywordstyle=\color{javapurple}\bfseries,
stringstyle=\color{javared},
commentstyle=\color{javagreen},
morecomment=[s][\color{javadocblue}]{/**}{*/},
numbers=left,
numberstyle=\tiny\color{black},
stepnumber=2,
numbersep=10pt,
tabsize=4,
showspaces=false,
frame=single,                  % adds a frame around the code
showstringspaces=false}

%\begin{frame}
%\frametitle{frame title} 
%\begin{center}
%\includegraphics[width=0.8\paperwidth]{images/figure.png}
%\end{center}
%\end{frame}

\begin{frame}[fragile]
\frametitle{Переменные и типы данных}
\framesubtitle{Применение}
\begin{enumerate}
  \item Шаблон:
    \newlineтип идентификатор [=значение][, идентификатор [=значение] ...];
    \newlineтип идентификатор[];
  \item Область видимости переменных
\end{enumerate}
\begin{lstlisting}
    byte hundred = 100;
    short thousand = 1000;
    int studentCount = 30;
    long humanCount = 7.0E9;

    float degree = 1.2;
    double weight = 67.5;

    char ch = 'J';

    boolean b = true;
\end{lstlisting}
\end{frame}

\begin{frame}[fragile]
\frametitle{Переменные и типы данных}
\framesubtitle{Массивы}
\begin{enumerate}
  \item Шаблон:
    \newlineтип идентификатор[ ] [= new тип [размер]];
\end{enumerate}
\begin{lstlisting}
    int days[];
    days = new int[30];
    days[0] = 28;

    int days[] = new int[30];

    int year[][] = new int[12][31];

    int houses[], cars[], wifes[];
    int[] houses, cars, wifes;
\end{lstlisting}
\end{frame}

\begin{frame}[fragile]
\frametitle{Операции}
\begin{itemize}
  \item<1-> Арифметические: + - * / \% ++ += -= *= /= \%= $--$
  \item<2-> Поразрядные: $\sim$ \& | \^{} $>>$ $>>>$ $<<$ \&= |= \^{}= $>>=$ $>>>=$ $<<=$
  \item<3-> Отношения: == != > < >= <=
  \item<4-> Логические: \& | \^{} || \&\& ! \&= |= \^{}= == != ?:
\end{itemize}
\end{frame}

\begin{frame}[fragile]
\frametitle{Операторы}
\framesubtitle{Условный оператор}
\begin{enumerate}
  \item Шаблон:
    \newline if (условие) инструкция1;
    \newline else if (условие) инструкция2;
    \newline else инструкция3;
\end{enumerate}
\begin{lstlisting}
    if (studentsCount == 30)
        deductSomeone();
    else if (studentsCount < 10)
        enrollSomeone();
\end{lstlisting}
\end{frame}

\begin{frame}[fragile]
\frametitle{Операторы}
\framesubtitle{Оператор ветвления}
\begin{enumerate}
  \item Шаблон:
    \newline switch (выражение) \{
    \newline     case значение1: инструкции; break;
    \newline     case значение2: инструкции; break;
    \newline     ...
    \newline     default: инструкции по умолчанию
    \newline \}
\end{enumerate}
\begin{lstlisting}
    switch (mark) {
        case 10:
            System.out.println("Excellent!");
            break;
        default:
            System.out.println("You can go hard or ...");
    }
\end{lstlisting}
\end{frame}

\begin{frame}[fragile]
\frametitle{Операторы}
\framesubtitle{Циклы}
\begin{enumerate}
  \item while:
    \newline while (условие) \{
    \newline     инструкции
    \newline \}
  \item do-while:
    \newline do \{
    \newline     инструкции
    \newline \} while (условие);
  \item for:
    \newline for (инициализация; условие; итерация) \{
    \newline     инструкции
    \newline \}
  \item for each:
    \newline for (тип переменная : коллекция) \{
    \newline     инструкции
    \newline \}
\end{enumerate}
\end{frame}

\end{document}
